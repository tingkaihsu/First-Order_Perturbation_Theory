\documentclass[12pt]{article}
\usepackage{braket}
\usepackage{physics}
\usepackage{graphicx}
\usepackage{times}
\usepackage[export]{adjustbox}
\usepackage{listings}
\usepackage{mathcomp}
\usepackage{hyperref}
\usepackage{bm,amsmath}
\usepackage{float}
\usepackage{indentfirst}
\usepackage{bigints}
\usepackage{listings}
\usepackage{color}
\hypersetup{
colorlinks=true,
linkcolor=blue,
filecolor=magenta,
urlcolor=cyan,
pdftitle={Overleaf Example},
pdfpagemode=FullScreen,
}
\definecolor{dkgreen}{rgb}{0,0.6,0}
\definecolor{gray}{rgb}{0.5,0.5,0.5}
\definecolor{mauve}{rgb}{0.58,0,0.82}
\lstset{frame=tb,
language=Python,
aboveskip=3mm,
belowskip=3mm,
stepnumber = 1,
showstringspaces=false,
columns=flexible,
basicstyle={\small\ttfamily},
numbers=left,
numberstyle=\color{gray},
keywordstyle=\color{blue},
commentstyle=\color{dkgreen},
stringstyle=\color{mauve},
breaklines=true,
breakatwhitespace=true,
tabsize=3
}
\numberwithin{equation}{section}

\title{First-Order Perturbation Theory}
\author{Ting-Kai Hsu}
\date{\today}

\begin{document}
\maketitle
\tableofcontents
\section{Degeneracy Problem}
Given a system that currently be regraded to have no interaction term in Hamiltonian, denoted as $H_0$.
Physical choice of representing the state of the system would be the eigenstates $\Psi_{a}$ of free-Hamiltonian with eigenvalues $E_{a}$, the eigen equation is the so-called time-independent Schr$\ddot{\text{o}}$dinger's equation.
\begin{equation}
    H_0\Psi_{a} = E_{a}\Psi_{a}
\end{equation}
\section{Wigner-Eckart theorem}
\end{document}